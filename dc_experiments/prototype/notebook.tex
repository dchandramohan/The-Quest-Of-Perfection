
% Default to the notebook output style

    


% Inherit from the specified cell style.




    
\documentclass[11pt]{article}

    
    
    \usepackage[T1]{fontenc}
    % Nicer default font (+ math font) than Computer Modern for most use cases
    \usepackage{mathpazo}

    % Basic figure setup, for now with no caption control since it's done
    % automatically by Pandoc (which extracts ![](path) syntax from Markdown).
    \usepackage{graphicx}
    % We will generate all images so they have a width \maxwidth. This means
    % that they will get their normal width if they fit onto the page, but
    % are scaled down if they would overflow the margins.
    \makeatletter
    \def\maxwidth{\ifdim\Gin@nat@width>\linewidth\linewidth
    \else\Gin@nat@width\fi}
    \makeatother
    \let\Oldincludegraphics\includegraphics
    % Set max figure width to be 80% of text width, for now hardcoded.
    \renewcommand{\includegraphics}[1]{\Oldincludegraphics[width=.8\maxwidth]{#1}}
    % Ensure that by default, figures have no caption (until we provide a
    % proper Figure object with a Caption API and a way to capture that
    % in the conversion process - todo).
    \usepackage{caption}
    \DeclareCaptionLabelFormat{nolabel}{}
    \captionsetup{labelformat=nolabel}

    \usepackage{adjustbox} % Used to constrain images to a maximum size 
    \usepackage{xcolor} % Allow colors to be defined
    \usepackage{enumerate} % Needed for markdown enumerations to work
    \usepackage{geometry} % Used to adjust the document margins
    \usepackage{amsmath} % Equations
    \usepackage{amssymb} % Equations
    \usepackage{textcomp} % defines textquotesingle
    % Hack from http://tex.stackexchange.com/a/47451/13684:
    \AtBeginDocument{%
        \def\PYZsq{\textquotesingle}% Upright quotes in Pygmentized code
    }
    \usepackage{upquote} % Upright quotes for verbatim code
    \usepackage{eurosym} % defines \euro
    \usepackage[mathletters]{ucs} % Extended unicode (utf-8) support
    \usepackage[utf8x]{inputenc} % Allow utf-8 characters in the tex document
    \usepackage{fancyvrb} % verbatim replacement that allows latex
    \usepackage{grffile} % extends the file name processing of package graphics 
                         % to support a larger range 
    % The hyperref package gives us a pdf with properly built
    % internal navigation ('pdf bookmarks' for the table of contents,
    % internal cross-reference links, web links for URLs, etc.)
    \usepackage{hyperref}
    \usepackage{longtable} % longtable support required by pandoc >1.10
    \usepackage{booktabs}  % table support for pandoc > 1.12.2
    \usepackage[inline]{enumitem} % IRkernel/repr support (it uses the enumerate* environment)
    \usepackage[normalem]{ulem} % ulem is needed to support strikethroughs (\sout)
                                % normalem makes italics be italics, not underlines
    

    
    
    % Colors for the hyperref package
    \definecolor{urlcolor}{rgb}{0,.145,.698}
    \definecolor{linkcolor}{rgb}{.71,0.21,0.01}
    \definecolor{citecolor}{rgb}{.12,.54,.11}

    % ANSI colors
    \definecolor{ansi-black}{HTML}{3E424D}
    \definecolor{ansi-black-intense}{HTML}{282C36}
    \definecolor{ansi-red}{HTML}{E75C58}
    \definecolor{ansi-red-intense}{HTML}{B22B31}
    \definecolor{ansi-green}{HTML}{00A250}
    \definecolor{ansi-green-intense}{HTML}{007427}
    \definecolor{ansi-yellow}{HTML}{DDB62B}
    \definecolor{ansi-yellow-intense}{HTML}{B27D12}
    \definecolor{ansi-blue}{HTML}{208FFB}
    \definecolor{ansi-blue-intense}{HTML}{0065CA}
    \definecolor{ansi-magenta}{HTML}{D160C4}
    \definecolor{ansi-magenta-intense}{HTML}{A03196}
    \definecolor{ansi-cyan}{HTML}{60C6C8}
    \definecolor{ansi-cyan-intense}{HTML}{258F8F}
    \definecolor{ansi-white}{HTML}{C5C1B4}
    \definecolor{ansi-white-intense}{HTML}{A1A6B2}

    % commands and environments needed by pandoc snippets
    % extracted from the output of `pandoc -s`
    \providecommand{\tightlist}{%
      \setlength{\itemsep}{0pt}\setlength{\parskip}{0pt}}
    \DefineVerbatimEnvironment{Highlighting}{Verbatim}{commandchars=\\\{\}}
    % Add ',fontsize=\small' for more characters per line
    \newenvironment{Shaded}{}{}
    \newcommand{\KeywordTok}[1]{\textcolor[rgb]{0.00,0.44,0.13}{\textbf{{#1}}}}
    \newcommand{\DataTypeTok}[1]{\textcolor[rgb]{0.56,0.13,0.00}{{#1}}}
    \newcommand{\DecValTok}[1]{\textcolor[rgb]{0.25,0.63,0.44}{{#1}}}
    \newcommand{\BaseNTok}[1]{\textcolor[rgb]{0.25,0.63,0.44}{{#1}}}
    \newcommand{\FloatTok}[1]{\textcolor[rgb]{0.25,0.63,0.44}{{#1}}}
    \newcommand{\CharTok}[1]{\textcolor[rgb]{0.25,0.44,0.63}{{#1}}}
    \newcommand{\StringTok}[1]{\textcolor[rgb]{0.25,0.44,0.63}{{#1}}}
    \newcommand{\CommentTok}[1]{\textcolor[rgb]{0.38,0.63,0.69}{\textit{{#1}}}}
    \newcommand{\OtherTok}[1]{\textcolor[rgb]{0.00,0.44,0.13}{{#1}}}
    \newcommand{\AlertTok}[1]{\textcolor[rgb]{1.00,0.00,0.00}{\textbf{{#1}}}}
    \newcommand{\FunctionTok}[1]{\textcolor[rgb]{0.02,0.16,0.49}{{#1}}}
    \newcommand{\RegionMarkerTok}[1]{{#1}}
    \newcommand{\ErrorTok}[1]{\textcolor[rgb]{1.00,0.00,0.00}{\textbf{{#1}}}}
    \newcommand{\NormalTok}[1]{{#1}}
    
    % Additional commands for more recent versions of Pandoc
    \newcommand{\ConstantTok}[1]{\textcolor[rgb]{0.53,0.00,0.00}{{#1}}}
    \newcommand{\SpecialCharTok}[1]{\textcolor[rgb]{0.25,0.44,0.63}{{#1}}}
    \newcommand{\VerbatimStringTok}[1]{\textcolor[rgb]{0.25,0.44,0.63}{{#1}}}
    \newcommand{\SpecialStringTok}[1]{\textcolor[rgb]{0.73,0.40,0.53}{{#1}}}
    \newcommand{\ImportTok}[1]{{#1}}
    \newcommand{\DocumentationTok}[1]{\textcolor[rgb]{0.73,0.13,0.13}{\textit{{#1}}}}
    \newcommand{\AnnotationTok}[1]{\textcolor[rgb]{0.38,0.63,0.69}{\textbf{\textit{{#1}}}}}
    \newcommand{\CommentVarTok}[1]{\textcolor[rgb]{0.38,0.63,0.69}{\textbf{\textit{{#1}}}}}
    \newcommand{\VariableTok}[1]{\textcolor[rgb]{0.10,0.09,0.49}{{#1}}}
    \newcommand{\ControlFlowTok}[1]{\textcolor[rgb]{0.00,0.44,0.13}{\textbf{{#1}}}}
    \newcommand{\OperatorTok}[1]{\textcolor[rgb]{0.40,0.40,0.40}{{#1}}}
    \newcommand{\BuiltInTok}[1]{{#1}}
    \newcommand{\ExtensionTok}[1]{{#1}}
    \newcommand{\PreprocessorTok}[1]{\textcolor[rgb]{0.74,0.48,0.00}{{#1}}}
    \newcommand{\AttributeTok}[1]{\textcolor[rgb]{0.49,0.56,0.16}{{#1}}}
    \newcommand{\InformationTok}[1]{\textcolor[rgb]{0.38,0.63,0.69}{\textbf{\textit{{#1}}}}}
    \newcommand{\WarningTok}[1]{\textcolor[rgb]{0.38,0.63,0.69}{\textbf{\textit{{#1}}}}}
    
    
    % Define a nice break command that doesn't care if a line doesn't already
    % exist.
    \def\br{\hspace*{\fill} \\* }
    % Math Jax compatability definitions
    \def\gt{>}
    \def\lt{<}
    % Document parameters
    \title{Procedurally Generated Levels Experiment}
    
    
    

    % Pygments definitions
    
\makeatletter
\def\PY@reset{\let\PY@it=\relax \let\PY@bf=\relax%
    \let\PY@ul=\relax \let\PY@tc=\relax%
    \let\PY@bc=\relax \let\PY@ff=\relax}
\def\PY@tok#1{\csname PY@tok@#1\endcsname}
\def\PY@toks#1+{\ifx\relax#1\empty\else%
    \PY@tok{#1}\expandafter\PY@toks\fi}
\def\PY@do#1{\PY@bc{\PY@tc{\PY@ul{%
    \PY@it{\PY@bf{\PY@ff{#1}}}}}}}
\def\PY#1#2{\PY@reset\PY@toks#1+\relax+\PY@do{#2}}

\expandafter\def\csname PY@tok@w\endcsname{\def\PY@tc##1{\textcolor[rgb]{0.73,0.73,0.73}{##1}}}
\expandafter\def\csname PY@tok@c\endcsname{\let\PY@it=\textit\def\PY@tc##1{\textcolor[rgb]{0.25,0.50,0.50}{##1}}}
\expandafter\def\csname PY@tok@cp\endcsname{\def\PY@tc##1{\textcolor[rgb]{0.74,0.48,0.00}{##1}}}
\expandafter\def\csname PY@tok@k\endcsname{\let\PY@bf=\textbf\def\PY@tc##1{\textcolor[rgb]{0.00,0.50,0.00}{##1}}}
\expandafter\def\csname PY@tok@kp\endcsname{\def\PY@tc##1{\textcolor[rgb]{0.00,0.50,0.00}{##1}}}
\expandafter\def\csname PY@tok@kt\endcsname{\def\PY@tc##1{\textcolor[rgb]{0.69,0.00,0.25}{##1}}}
\expandafter\def\csname PY@tok@o\endcsname{\def\PY@tc##1{\textcolor[rgb]{0.40,0.40,0.40}{##1}}}
\expandafter\def\csname PY@tok@ow\endcsname{\let\PY@bf=\textbf\def\PY@tc##1{\textcolor[rgb]{0.67,0.13,1.00}{##1}}}
\expandafter\def\csname PY@tok@nb\endcsname{\def\PY@tc##1{\textcolor[rgb]{0.00,0.50,0.00}{##1}}}
\expandafter\def\csname PY@tok@nf\endcsname{\def\PY@tc##1{\textcolor[rgb]{0.00,0.00,1.00}{##1}}}
\expandafter\def\csname PY@tok@nc\endcsname{\let\PY@bf=\textbf\def\PY@tc##1{\textcolor[rgb]{0.00,0.00,1.00}{##1}}}
\expandafter\def\csname PY@tok@nn\endcsname{\let\PY@bf=\textbf\def\PY@tc##1{\textcolor[rgb]{0.00,0.00,1.00}{##1}}}
\expandafter\def\csname PY@tok@ne\endcsname{\let\PY@bf=\textbf\def\PY@tc##1{\textcolor[rgb]{0.82,0.25,0.23}{##1}}}
\expandafter\def\csname PY@tok@nv\endcsname{\def\PY@tc##1{\textcolor[rgb]{0.10,0.09,0.49}{##1}}}
\expandafter\def\csname PY@tok@no\endcsname{\def\PY@tc##1{\textcolor[rgb]{0.53,0.00,0.00}{##1}}}
\expandafter\def\csname PY@tok@nl\endcsname{\def\PY@tc##1{\textcolor[rgb]{0.63,0.63,0.00}{##1}}}
\expandafter\def\csname PY@tok@ni\endcsname{\let\PY@bf=\textbf\def\PY@tc##1{\textcolor[rgb]{0.60,0.60,0.60}{##1}}}
\expandafter\def\csname PY@tok@na\endcsname{\def\PY@tc##1{\textcolor[rgb]{0.49,0.56,0.16}{##1}}}
\expandafter\def\csname PY@tok@nt\endcsname{\let\PY@bf=\textbf\def\PY@tc##1{\textcolor[rgb]{0.00,0.50,0.00}{##1}}}
\expandafter\def\csname PY@tok@nd\endcsname{\def\PY@tc##1{\textcolor[rgb]{0.67,0.13,1.00}{##1}}}
\expandafter\def\csname PY@tok@s\endcsname{\def\PY@tc##1{\textcolor[rgb]{0.73,0.13,0.13}{##1}}}
\expandafter\def\csname PY@tok@sd\endcsname{\let\PY@it=\textit\def\PY@tc##1{\textcolor[rgb]{0.73,0.13,0.13}{##1}}}
\expandafter\def\csname PY@tok@si\endcsname{\let\PY@bf=\textbf\def\PY@tc##1{\textcolor[rgb]{0.73,0.40,0.53}{##1}}}
\expandafter\def\csname PY@tok@se\endcsname{\let\PY@bf=\textbf\def\PY@tc##1{\textcolor[rgb]{0.73,0.40,0.13}{##1}}}
\expandafter\def\csname PY@tok@sr\endcsname{\def\PY@tc##1{\textcolor[rgb]{0.73,0.40,0.53}{##1}}}
\expandafter\def\csname PY@tok@ss\endcsname{\def\PY@tc##1{\textcolor[rgb]{0.10,0.09,0.49}{##1}}}
\expandafter\def\csname PY@tok@sx\endcsname{\def\PY@tc##1{\textcolor[rgb]{0.00,0.50,0.00}{##1}}}
\expandafter\def\csname PY@tok@m\endcsname{\def\PY@tc##1{\textcolor[rgb]{0.40,0.40,0.40}{##1}}}
\expandafter\def\csname PY@tok@gh\endcsname{\let\PY@bf=\textbf\def\PY@tc##1{\textcolor[rgb]{0.00,0.00,0.50}{##1}}}
\expandafter\def\csname PY@tok@gu\endcsname{\let\PY@bf=\textbf\def\PY@tc##1{\textcolor[rgb]{0.50,0.00,0.50}{##1}}}
\expandafter\def\csname PY@tok@gd\endcsname{\def\PY@tc##1{\textcolor[rgb]{0.63,0.00,0.00}{##1}}}
\expandafter\def\csname PY@tok@gi\endcsname{\def\PY@tc##1{\textcolor[rgb]{0.00,0.63,0.00}{##1}}}
\expandafter\def\csname PY@tok@gr\endcsname{\def\PY@tc##1{\textcolor[rgb]{1.00,0.00,0.00}{##1}}}
\expandafter\def\csname PY@tok@ge\endcsname{\let\PY@it=\textit}
\expandafter\def\csname PY@tok@gs\endcsname{\let\PY@bf=\textbf}
\expandafter\def\csname PY@tok@gp\endcsname{\let\PY@bf=\textbf\def\PY@tc##1{\textcolor[rgb]{0.00,0.00,0.50}{##1}}}
\expandafter\def\csname PY@tok@go\endcsname{\def\PY@tc##1{\textcolor[rgb]{0.53,0.53,0.53}{##1}}}
\expandafter\def\csname PY@tok@gt\endcsname{\def\PY@tc##1{\textcolor[rgb]{0.00,0.27,0.87}{##1}}}
\expandafter\def\csname PY@tok@err\endcsname{\def\PY@bc##1{\setlength{\fboxsep}{0pt}\fcolorbox[rgb]{1.00,0.00,0.00}{1,1,1}{\strut ##1}}}
\expandafter\def\csname PY@tok@kc\endcsname{\let\PY@bf=\textbf\def\PY@tc##1{\textcolor[rgb]{0.00,0.50,0.00}{##1}}}
\expandafter\def\csname PY@tok@kd\endcsname{\let\PY@bf=\textbf\def\PY@tc##1{\textcolor[rgb]{0.00,0.50,0.00}{##1}}}
\expandafter\def\csname PY@tok@kn\endcsname{\let\PY@bf=\textbf\def\PY@tc##1{\textcolor[rgb]{0.00,0.50,0.00}{##1}}}
\expandafter\def\csname PY@tok@kr\endcsname{\let\PY@bf=\textbf\def\PY@tc##1{\textcolor[rgb]{0.00,0.50,0.00}{##1}}}
\expandafter\def\csname PY@tok@bp\endcsname{\def\PY@tc##1{\textcolor[rgb]{0.00,0.50,0.00}{##1}}}
\expandafter\def\csname PY@tok@fm\endcsname{\def\PY@tc##1{\textcolor[rgb]{0.00,0.00,1.00}{##1}}}
\expandafter\def\csname PY@tok@vc\endcsname{\def\PY@tc##1{\textcolor[rgb]{0.10,0.09,0.49}{##1}}}
\expandafter\def\csname PY@tok@vg\endcsname{\def\PY@tc##1{\textcolor[rgb]{0.10,0.09,0.49}{##1}}}
\expandafter\def\csname PY@tok@vi\endcsname{\def\PY@tc##1{\textcolor[rgb]{0.10,0.09,0.49}{##1}}}
\expandafter\def\csname PY@tok@vm\endcsname{\def\PY@tc##1{\textcolor[rgb]{0.10,0.09,0.49}{##1}}}
\expandafter\def\csname PY@tok@sa\endcsname{\def\PY@tc##1{\textcolor[rgb]{0.73,0.13,0.13}{##1}}}
\expandafter\def\csname PY@tok@sb\endcsname{\def\PY@tc##1{\textcolor[rgb]{0.73,0.13,0.13}{##1}}}
\expandafter\def\csname PY@tok@sc\endcsname{\def\PY@tc##1{\textcolor[rgb]{0.73,0.13,0.13}{##1}}}
\expandafter\def\csname PY@tok@dl\endcsname{\def\PY@tc##1{\textcolor[rgb]{0.73,0.13,0.13}{##1}}}
\expandafter\def\csname PY@tok@s2\endcsname{\def\PY@tc##1{\textcolor[rgb]{0.73,0.13,0.13}{##1}}}
\expandafter\def\csname PY@tok@sh\endcsname{\def\PY@tc##1{\textcolor[rgb]{0.73,0.13,0.13}{##1}}}
\expandafter\def\csname PY@tok@s1\endcsname{\def\PY@tc##1{\textcolor[rgb]{0.73,0.13,0.13}{##1}}}
\expandafter\def\csname PY@tok@mb\endcsname{\def\PY@tc##1{\textcolor[rgb]{0.40,0.40,0.40}{##1}}}
\expandafter\def\csname PY@tok@mf\endcsname{\def\PY@tc##1{\textcolor[rgb]{0.40,0.40,0.40}{##1}}}
\expandafter\def\csname PY@tok@mh\endcsname{\def\PY@tc##1{\textcolor[rgb]{0.40,0.40,0.40}{##1}}}
\expandafter\def\csname PY@tok@mi\endcsname{\def\PY@tc##1{\textcolor[rgb]{0.40,0.40,0.40}{##1}}}
\expandafter\def\csname PY@tok@il\endcsname{\def\PY@tc##1{\textcolor[rgb]{0.40,0.40,0.40}{##1}}}
\expandafter\def\csname PY@tok@mo\endcsname{\def\PY@tc##1{\textcolor[rgb]{0.40,0.40,0.40}{##1}}}
\expandafter\def\csname PY@tok@ch\endcsname{\let\PY@it=\textit\def\PY@tc##1{\textcolor[rgb]{0.25,0.50,0.50}{##1}}}
\expandafter\def\csname PY@tok@cm\endcsname{\let\PY@it=\textit\def\PY@tc##1{\textcolor[rgb]{0.25,0.50,0.50}{##1}}}
\expandafter\def\csname PY@tok@cpf\endcsname{\let\PY@it=\textit\def\PY@tc##1{\textcolor[rgb]{0.25,0.50,0.50}{##1}}}
\expandafter\def\csname PY@tok@c1\endcsname{\let\PY@it=\textit\def\PY@tc##1{\textcolor[rgb]{0.25,0.50,0.50}{##1}}}
\expandafter\def\csname PY@tok@cs\endcsname{\let\PY@it=\textit\def\PY@tc##1{\textcolor[rgb]{0.25,0.50,0.50}{##1}}}

\def\PYZbs{\char`\\}
\def\PYZus{\char`\_}
\def\PYZob{\char`\{}
\def\PYZcb{\char`\}}
\def\PYZca{\char`\^}
\def\PYZam{\char`\&}
\def\PYZlt{\char`\<}
\def\PYZgt{\char`\>}
\def\PYZsh{\char`\#}
\def\PYZpc{\char`\%}
\def\PYZdl{\char`\$}
\def\PYZhy{\char`\-}
\def\PYZsq{\char`\'}
\def\PYZdq{\char`\"}
\def\PYZti{\char`\~}
% for compatibility with earlier versions
\def\PYZat{@}
\def\PYZlb{[}
\def\PYZrb{]}
\makeatother


    % Exact colors from NB
    \definecolor{incolor}{rgb}{0.0, 0.0, 0.5}
    \definecolor{outcolor}{rgb}{0.545, 0.0, 0.0}



    
    % Prevent overflowing lines due to hard-to-break entities
    \sloppy 
    % Setup hyperref package
    \hypersetup{
      breaklinks=true,  % so long urls are correctly broken across lines
      colorlinks=true,
      urlcolor=urlcolor,
      linkcolor=linkcolor,
      citecolor=citecolor,
      }
    % Slightly bigger margins than the latex defaults
    
    \geometry{verbose,tmargin=1in,bmargin=1in,lmargin=1in,rmargin=1in}
    
    

    \begin{document}
    
    
    \maketitle
    
    

    
    \hypertarget{the-quest-of-perfection-prototype}{%
\section{The Quest of Perfection
(Prototype)}\label{the-quest-of-perfection-prototype}}

\hypertarget{drawing-a-scene}{%
\subsection{Drawing a scene}\label{drawing-a-scene}}

I'm testing out procedurally generated levels in this notebook. I
generated a couple textures and now I'm going to try to put them
together into a scene!

    \begin{Verbatim}[commandchars=\\\{\}]
{\color{incolor}In [{\color{incolor}1}]:} \PY{c+c1}{\PYZsh{} first import useful packages}
        \PY{k+kn}{from} \PY{n+nn}{ipywidgets} \PY{k}{import} \PY{n}{Image}\PY{p}{,} \PY{n}{HBox} \PY{c+c1}{\PYZsh{} \PYZdq{}HBox\PYZdq{} is for a sanity check}
        \PY{k+kn}{from} \PY{n+nn}{ipycanvas} \PY{k}{import} \PY{n}{Canvas}\PY{p}{,} \PY{n}{hold\PYZus{}canvas}
\end{Verbatim}


    I'm using the
\href{https://ipycanvas.readthedocs.io/en/latest/index.html}{ipycanvas
(Interactive Canvas in Jupyter)} package to make manipulating the images
easier\ldots{}

    \hypertarget{load-the-images}{%
\subsection{Load the images}\label{load-the-images}}

Load the texture/sprite images I generated and collect some of their
properties\ldots{}

    \begin{Verbatim}[commandchars=\\\{\}]
{\color{incolor}In [{\color{incolor}2}]:} \PY{n}{im\PYZus{}Sprite} \PY{o}{=} \PY{n}{Image}\PY{o}{.}\PY{n}{from\PYZus{}file}\PY{p}{(}\PY{l+s+s1}{\PYZsq{}}\PY{l+s+s1}{img/sprite0.png}\PY{l+s+s1}{\PYZsq{}}\PY{p}{)}
        
        \PY{n}{im\PYZus{}txGrnd} \PY{o}{=} \PY{n}{Image}\PY{o}{.}\PY{n}{from\PYZus{}file}\PY{p}{(}\PY{l+s+s1}{\PYZsq{}}\PY{l+s+s1}{img/texture\PYZus{}Ground.png}\PY{l+s+s1}{\PYZsq{}}\PY{p}{)}
        \PY{n}{im\PYZus{}txStar} \PY{o}{=} \PY{n}{Image}\PY{o}{.}\PY{n}{from\PYZus{}file}\PY{p}{(}\PY{l+s+s1}{\PYZsq{}}\PY{l+s+s1}{img/texture\PYZus{}Gem.png}\PY{l+s+s1}{\PYZsq{}}\PY{p}{)}
\end{Verbatim}


    \begin{Verbatim}[commandchars=\\\{\}]
{\color{incolor}In [{\color{incolor}3}]:} \PY{n}{im\PYZus{}Sprite}\PY{o}{.}\PY{n+nv+vm}{\PYZus{}\PYZus{}dict\PYZus{}\PYZus{}}
\end{Verbatim}


\begin{Verbatim}[commandchars=\\\{\}]
{\color{outcolor}Out[{\color{outcolor}3}]:} \{'\_trait\_values': \{'\_model\_module': '@jupyter-widgets/controls',
          '\_model\_module\_version': '1.5.0',
          '\_model\_name': 'ImageModel',
          '\_view\_count': None,
          '\_view\_module': '@jupyter-widgets/controls',
          '\_view\_module\_version': '1.5.0',
          '\_view\_name': 'ImageView',
          'format': 'png',
          'height': '',
          'value': b'\textbackslash{}x89PNG\textbackslash{}r\textbackslash{}n\textbackslash{}x1a\textbackslash{}n\textbackslash{}x00\textbackslash{}x00\textbackslash{}x00\textbackslash{}rIHDR\textbackslash{}x00\textbackslash{}x00\textbackslash{}x00 \textbackslash{}x00\textbackslash{}x00\textbackslash{}x000\textbackslash{}x08\textbackslash{}x06\textbackslash{}x00\textbackslash{}x00\textbackslash{}x00p\textbackslash{}xacxo\textbackslash{}x00\textbackslash{}x00\textbackslash{}x00\textbackslash{}x06bKGD\textbackslash{}x00\textbackslash{}x85\textbackslash{}x00Q\textbackslash{}x00(\textbackslash{}xd8\textbackslash{}x8eh\textbackslash{}xe0\textbackslash{}x00\textbackslash{}x00\textbackslash{}x00\textbackslash{}tpHYs\textbackslash{}x00\textbackslash{}x00.\#\textbackslash{}x00\textbackslash{}x00.\#\textbackslash{}x01x\textbackslash{}xa5?v\textbackslash{}x00\textbackslash{}x00\textbackslash{}x00\textbackslash{}x07tIME\textbackslash{}x07\textbackslash{}xe3\textbackslash{}x04\textbackslash{}x01\textbackslash{}x07\textbackslash{}x0b4\textbackslash{}xd679\textbackslash{}xd2\textbackslash{}x00\textbackslash{}x00\textbackslash{}x00\textbackslash{}x19tEXtComment\textbackslash{}x00Created with GIMPW\textbackslash{}x81\textbackslash{}x0e\textbackslash{}x17\textbackslash{}x00\textbackslash{}x00\textbackslash{}x01\textbackslash{}xb0IDATX\textbackslash{}xc3\textbackslash{}xed\textbackslash{}x981n\textbackslash{}xc30\textbackslash{}x0cEI\textbackslash{}xc1\textbackslash{}xcd\textbackslash{}x01z\textbackslash{}x81"\textbackslash{}xc8\textbackslash{}x1e\textbackslash{}xf8\textbackslash{}x0c>\textbackslash{}x84\textbackslash{}xa7\textbackslash{}x9e"[;\textbackslash{}x17\textbackslash{}x1ds\textbackslash{}x10\textbackslash{}x1f"g\textbackslash{}x10\textbackslash{}xbc\textbackslash{}x1bA\textbackslash{}x8f\textbackslash{}xd0\textbackslash{}xd5\textbackslash{}x1d\textbackslash{}xd4\textbackslash{}xa1V\textbackslash{}xa1(\textbackslash{}x94D1\textbackslash{}xb2k \textbackslash{}x15\textbackslash{}x10\$\textbackslash{}x96L\textbackslash{}xbe/\textbackslash{}x9a\textbackslash{}xa2\textbackslash{}x14\textbackslash{}x03\textbackslash{}xdc\{C\textbackslash{}xa1\textbackslash{}x9d)\textbackslash{}xe5\textbackslash{}x0f\textbackslash{}x0b@o\textbackslash{}xf2\textbackslash{}x8d\textbackslash{}x05\textbackslash{}x81"\textbackslash{}x16\textbackslash{}xce\textbackslash{}x05>4\textbackslash{}xdb\textbackslash{}xdf\textbackslash{}xdf\textbackslash{}xc7\textbackslash{}xd39\textbackslash{}xc8\textbackslash{}xc4Rp\textbackslash{}x17\textbackslash{}x18\textbackslash{}x00\textbackslash{}x93\textbackslash{}xdc\textbackslash{}xaa\textbackslash{}xe4\textbackslash{}xcc\#@*\textbackslash{}xca\textbackslash{}x08\textbackslash{}x00\textbackslash{}xa0\textbackslash{}xe6\textbackslash{}n;7\textbackslash{}xa91W@\&\$\textbackslash{}x15\textbackslash{}x1d\textbackslash{}x8c\textbackslash{}n\textbackslash{}x90\textbackslash{}xc2|h\textbackslash{}xc8\textbackslash{}xcf\textbackslash{}xf1t\textbackslash{}xfe\textbackslash{}x11ph\textbackslash{}xb6F\textbackslash{}xea\textbackslash{}x84;\textbackslash{}xe3\textbackslash{}xc0\textbackslash{}xaa@e;\textbackslash{}xa8\textbackslash{}x10\textbackslash{}xb9\}\textbackslash{}x19\textbackslash{}tv\textbackslash{}x05\textbackslash{}x8c\textbackslash{}x89W\%\textbackslash{}x9c\textbackslash{}xc4\textbackslash{}xecSvH-\textbackslash{}xc3\textbackslash{}xc83\textbackslash{}x13\textbackslash{}xaf\textbackslash{}x08\textbackslash{}xc2\textbackslash{}x16\textbackslash{}xa3\textbackslash{}x85\textbackslash{}xc8:\textbackslash{}xc8\textbackslash{}r\}nY\textbackslash{}x9e\textbackslash{}xad\textbackslash{}x14s\textbackslash{}xf7\textbackslash{}x04\textbackslash{}xcc\textbackslash{}xd9\textbackslash{}x88\textbackslash{}xb8y\textbackslash{}xc0)\textbackslash{}xc1\textbackslash{}xdc-3KH\textbackslash{}x0e\textbackslash{}x98\textbackslash{}xdd\textbackslash{}xea\textbackslash{}xba6\textbackslash{}x93\textbackslash{}x90\textbackslash{}xec\textbackslash{}xcfd+;\textbackslash{}x0fP\textbackslash{}xc6Zk\textbackslash{}xaeh l\textbackslash{}xd9\textbackslash{}xe7\textbackslash{}x01\textbackslash{}xe0(\textbackslash{}x974J\textbackslash{}x04r\textbackslash{}xe1\textbackslash{}xef\_\textbackslash{}x1f\textbackslash{}x17\textbackslash{}xd7\textbackslash{}xaf\textbackslash{}x0fOY\textbackslash{}xe3!\textbackslash{}x11(\textbackslash{}x99\textbackslash{}xb9\textbackslash{}x85Y\textbackslash{}x08u\textbackslash{}x1d\textbackslash{}x12\textbackslash{}xe0\textbackslash{}x8bP\textbackslash{}\textbackslash{}\textbackslash{}xb8;C\textbackslash{}xca\textbackslash{}xb9\textbackslash{}xdb\textbackslash{}x17\textbackslash{}x83\textbackslash{}xfb,\%y\textbackslash{}x96\textasciitilde{}\textbackslash{}xb8\textbackslash{}xfd>j<\textbackslash{}xba\textbackslash{}nB\textbackslash{}xb3\textbackslash{}xd7Z\textbackslash{}xa3\textbackslash{}x1d\textbackslash{}x93\textbackslash{}xe6\textbackslash{}x80\textbackslash{}xeb\textbackslash{}x83\textbackslash{}xf2\_q\textbackslash{}x95\textbackslash{}xa6\textbackslash{}xc2\textbackslash{}x9a\textbackslash{}x1a\textbackslash{}x17m\textbackslash{}xc7K\textbackslash{}xb4\textbackslash{}x8a\textbackslash{}xb1v\textbackslash{}x83c]\textbackslash{}xd7\textbackslash{}x01\textbackslash{}x00@\textbackslash{}xdb\textbackslash{}xb6b\textbackslash{}x01*\textbackslash{}xa7h\textbackslash{}x94,@\textbackslash{}x17\textbackslash{}x11h\textbackslash{}x1e?\textbackslash{}xaf\textbackslash{}x8d\textbackslash{}xa6\textbackslash{}xef\textbackslash{}xcdn\textbackslash{}x1ft\textbackslash{}xfc\textbackslash{}xfc\textbackslash{}xf2\textbackslash{}x96\textbackslash{}xbcg\textbackslash{}x1cz\textbackslash{}x881\textbackslash{}xfe<\textbackslash{}x07\textbackslash{}xec\textbackslash{}xa1\textbackslash{}x14\textbackslash{}xfd\textbackslash{}xe32\textbackslash{}xc7x\textbackslash{}x1cz\textbackslash{}x1c\textbackslash{}x87\textbackslash{}x9euo\textbackslash{}x88\textbackslash{}xa1r\textbackslash{}xa1\%\textbackslash{}x9a\textbackslash{}xcbJB7\textbackslash{}xbb\textbackslash{}xbd\textbackslash{}x89E\textbackslash{}x80\{\textbackslash{}xcf-\textbackslash{}'\textbackslash{}x15\textbackslash{}xc3\textbackslash{}xb07R\textbackslash{}xc6:\textbackslash{}x92\textbackslash{}xf0\textbackslash{}xae\textbackslash{}x05T\textbackslash{}x0b\textbackslash{}xbc\#2\textbackslash{}xb1\textbackslash{}\textbackslash{}P\textbackslash{}x0b\textbackslash{}xbd\textbackslash{}xa0\textbackslash{}n\textbackslash{}xfaT\textbackslash{}x0b\textbackslash{}xff32\textbackslash{}xabOB\textbackslash{}xb3\textbackslash{}x00\textbackslash{}xd3\textbackslash{}xfc/\textbackslash{}xc3U\textbackslash{}t\textbackslash{}xf8\textbackslash{}x06s\textbackslash{}xcb\textbackslash{}xd5E\{F\textbackslash{}xc8\textbackslash{}x8e\textbackslash{}x00\textbackslash{}x00\textbackslash{}x00\textbackslash{}x00IEND\textbackslash{}xaeB`\textbackslash{}x82',
          'width': '',
          'comm': <ipykernel.comm.comm.Comm at 0x7f668fb0beb0>,
          'keys': ['\_dom\_classes',
           '\_model\_module',
           '\_model\_module\_version',
           '\_model\_name',
           '\_view\_count',
           '\_view\_module',
           '\_view\_module\_version',
           '\_view\_name',
           'format',
           'height',
           'layout',
           'value',
           'width'],
          '\_dom\_classes': (),
          'layout': Layout()\},
         '\_trait\_notifiers': \{'comm': \{'change': [<traitlets.traitlets.ObserveHandler at 0x7f66b42a6a00>]\}\},
         '\_trait\_validators': \{\},
         '\_cross\_validation\_lock': False,
         '\_model\_id': 'e84e3eec3c4c45c190308523c2ed3ba2'\}
\end{Verbatim}
            
    Ok\ldots{} so what I learned from the above is that the height, width,
etc, etc, info are not stored automatically when the images is loaded (I
think)\ldots{} Thankfully I know the dimensions of the images (I also
think), because I created them!

The following syntax is derived from the examples
\href{https://ipycanvas.readthedocs.io/en/latest/drawing_images.html}{here}.

    \begin{Verbatim}[commandchars=\\\{\}]
{\color{incolor}In [{\color{incolor}4}]:} \PY{c+c1}{\PYZsh{} hard\PYZhy{}coded width/height in px}
        \PY{c+c1}{\PYZsh{} ... I\PYZsq{}m adding a little padding to my sprite which is 32 x 48 px}
        \PY{n}{cSprite} \PY{o}{=} \PY{n}{Canvas}\PY{p}{(}\PY{n}{width}\PY{o}{=}\PY{l+m+mi}{36}\PY{p}{,} \PY{n}{height}\PY{o}{=}\PY{l+m+mi}{48}\PY{p}{)}
        \PY{n}{cTxGrnd} \PY{o}{=} \PY{n}{Canvas}\PY{p}{(}\PY{n}{width}\PY{o}{=}\PY{l+m+mi}{600}\PY{p}{,} \PY{n}{height}\PY{o}{=}\PY{l+m+mi}{124}\PY{p}{)}
        \PY{n}{cTxStar} \PY{o}{=} \PY{n}{Canvas}\PY{p}{(}\PY{n}{width}\PY{o}{=}\PY{l+m+mi}{136}\PY{p}{,} \PY{n}{height}\PY{o}{=}\PY{l+m+mi}{152}\PY{p}{)}
        
        \PY{n}{cSprite}\PY{o}{.}\PY{n}{draw\PYZus{}image}\PY{p}{(}\PY{n}{im\PYZus{}Sprite}\PY{p}{,} \PY{l+m+mi}{2}\PY{p}{,} \PY{l+m+mi}{0}\PY{p}{)}
        \PY{n}{cTxGrnd}\PY{o}{.}\PY{n}{draw\PYZus{}image}\PY{p}{(}\PY{n}{im\PYZus{}txGrnd}\PY{p}{,} \PY{l+m+mi}{0}\PY{p}{,} \PY{l+m+mi}{0}\PY{p}{)}
        \PY{n}{cTxStar}\PY{o}{.}\PY{n}{draw\PYZus{}image}\PY{p}{(}\PY{n}{im\PYZus{}txStar}\PY{p}{,} \PY{l+m+mi}{0}\PY{p}{,} \PY{l+m+mi}{0}\PY{p}{)}
        
        \PY{n}{cPlatform} \PY{o}{=} \PY{n}{Canvas}\PY{p}{(}\PY{n}{width}\PY{o}{=}\PY{l+m+mi}{100}\PY{p}{,} \PY{n}{height}\PY{o}{=}\PY{l+m+mi}{56}\PY{p}{)}
        \PY{n}{cPlatform}\PY{o}{.}\PY{n}{draw\PYZus{}image}\PY{p}{(}\PY{n}{cTxGrnd}\PY{p}{,} \PY{l+m+mi}{0}\PY{p}{,} \PY{l+m+mi}{0}\PY{p}{,} \PY{n}{width}\PY{o}{=}\PY{p}{(}\PY{l+m+mi}{56}\PY{o}{/}\PY{l+m+mi}{124}\PY{p}{)}\PY{o}{*}\PY{l+m+mi}{600}\PY{p}{,} \PY{n}{height}\PY{o}{=}\PY{l+m+mi}{56}\PY{p}{)}
        
        \PY{n}{cStar} \PY{o}{=} \PY{n}{Canvas}\PY{p}{(}\PY{n}{width}\PY{o}{=}\PY{l+m+mi}{42}\PY{p}{,} \PY{n}{height}\PY{o}{=}\PY{l+m+mi}{48}\PY{p}{)}
        \PY{n}{cStar}\PY{o}{.}\PY{n}{draw\PYZus{}image}\PY{p}{(}\PY{n}{cTxStar}\PY{p}{,} \PY{l+m+mi}{0}\PY{p}{,} \PY{l+m+mi}{0}\PY{p}{,} \PY{n}{width}\PY{o}{=}\PY{l+m+mi}{42}\PY{p}{,} \PY{n}{height}\PY{o}{=}\PY{l+m+mi}{48}\PY{p}{)}
        
        \PY{c+c1}{\PYZsh{} sanity check!}
        \PY{n}{HBox}\PY{p}{(}\PY{p}{[}\PY{n}{cSprite}\PY{p}{,}\PY{n}{cSprite}\PY{p}{,}\PY{n}{cSprite}\PY{p}{,}\PY{n}{cTxGrnd}\PY{p}{,}\PY{n}{cTxStar}\PY{p}{]}\PY{p}{)}
\end{Verbatim}


    
    \begin{verbatim}
HBox(children=(Canvas(height=48, width=36), Canvas(height=48, width=36), Canvas(height=48, width=36), Canvas(h…
    \end{verbatim}

    
    \begin{Verbatim}[commandchars=\\\{\}]
{\color{incolor}In [{\color{incolor}5}]:} \PY{k+kn}{from} \PY{n+nn}{random} \PY{k}{import} \PY{n}{randint}
        
        \PY{n}{cScene} \PY{o}{=} \PY{n}{Canvas}\PY{p}{(}\PY{n}{Width}\PY{o}{=}\PY{l+m+mi}{600}\PY{p}{,} \PY{n}{height}\PY{o}{=}\PY{l+m+mi}{600}\PY{p}{)}
        
        \PY{n}{cScene}\PY{o}{.}\PY{n}{fill\PYZus{}style} \PY{o}{=} \PY{l+s+s1}{\PYZsq{}}\PY{l+s+s1}{\PYZsh{}a9cafc}\PY{l+s+s1}{\PYZsq{}}
        \PY{n}{cScene}\PY{o}{.}\PY{n}{fill\PYZus{}rect}\PY{p}{(}\PY{l+m+mi}{0}\PY{p}{,} \PY{l+m+mi}{0}\PY{p}{,} \PY{l+m+mi}{600}\PY{p}{,} \PY{l+m+mi}{600}\PY{p}{)}
        
        \PY{n}{cScene}\PY{o}{.}\PY{n}{draw\PYZus{}image}\PY{p}{(}\PY{n}{cTxGrnd}\PY{p}{,} \PY{l+m+mi}{0}\PY{p}{,} \PY{l+m+mi}{600}\PY{o}{\PYZhy{}}\PY{l+m+mi}{124}\PY{p}{)}
        \PY{n}{cScene}\PY{o}{.}\PY{n}{draw\PYZus{}image}\PY{p}{(}\PY{n}{cSprite}\PY{p}{,} \PY{l+m+mi}{10}\PY{p}{,} \PY{l+m+mi}{600}\PY{o}{\PYZhy{}}\PY{p}{(}\PY{l+m+mi}{48}\PY{o}{+}\PY{l+m+mi}{124}\PY{p}{)}\PY{p}{)}
        
        \PY{n}{cScene}\PY{o}{.}\PY{n}{draw\PYZus{}image}\PY{p}{(}\PY{n}{cPlatform}\PY{p}{,} \PY{l+m+mi}{275}\PY{p}{,} \PY{l+m+mi}{250}\PY{p}{)}
        \PY{n}{cScene}\PY{o}{.}\PY{n}{draw\PYZus{}image}\PY{p}{(}\PY{n}{cStar}\PY{p}{,} 
                          \PY{l+m+mi}{275} \PY{o}{+} \PY{l+m+mi}{50} \PY{o}{\PYZhy{}} \PY{l+m+mi}{42}\PY{o}{/}\PY{l+m+mi}{2}\PY{p}{,} \PY{c+c1}{\PYZsh{} x\PYZhy{}pos: centered on platform}
                          \PY{l+m+mi}{250} \PY{o}{\PYZhy{}} \PY{l+m+mi}{50}\PY{p}{)} \PY{c+c1}{\PYZsh{} y\PYZhy{}pos: hovering 2 px over platform}
        
        \PY{n}{cScene}\PY{o}{.}\PY{n}{draw\PYZus{}image}\PY{p}{(}\PY{n}{cPlatform}\PY{p}{,} \PY{n}{randint}\PY{p}{(}\PY{l+m+mi}{50}\PY{p}{,}\PY{l+m+mi}{200}\PY{p}{)}\PY{p}{,} \PY{l+m+mi}{350}\PY{p}{)}
        \PY{n}{cScene}\PY{o}{.}\PY{n}{draw\PYZus{}image}\PY{p}{(}\PY{n}{cPlatform}\PY{p}{,} \PY{n}{randint}\PY{p}{(}\PY{l+m+mi}{300}\PY{p}{,}\PY{l+m+mi}{450}\PY{p}{)}\PY{p}{,} \PY{l+m+mi}{350}\PY{p}{)}
        
        \PY{n}{xtop} \PY{o}{=}  \PY{n}{randint}\PY{p}{(}\PY{l+m+mi}{25}\PY{p}{,}\PY{l+m+mi}{475}\PY{p}{)}
        \PY{n}{cScene}\PY{o}{.}\PY{n}{draw\PYZus{}image}\PY{p}{(}\PY{n}{cPlatform}\PY{p}{,} \PY{n}{xtop}\PY{p}{,} \PY{l+m+mi}{100}\PY{p}{)}
        \PY{n}{cScene}\PY{o}{.}\PY{n}{draw\PYZus{}image}\PY{p}{(}\PY{n}{cStar}\PY{p}{,} 
                          \PY{n}{xtop} \PY{o}{+} \PY{l+m+mi}{50} \PY{o}{\PYZhy{}} \PY{l+m+mi}{42}\PY{o}{/}\PY{l+m+mi}{2}\PY{p}{,} \PY{c+c1}{\PYZsh{} x\PYZhy{}pos: centered on platform}
                          \PY{l+m+mi}{100} \PY{o}{\PYZhy{}} \PY{l+m+mi}{50}\PY{p}{)} \PY{c+c1}{\PYZsh{} y\PYZhy{}pos: hovering 2 px over platform}
        
        \PY{n}{cScene}
\end{Verbatim}


    
    \begin{verbatim}
Canvas(height=600)
    \end{verbatim}

    
    If you run the previous cell a few times the blocks on the lowest level
and the blocks on the highest level change positions based on the random
value returned by \texttt{randint}!

So we've done it\ldots{} we've generated a basic procedurally generated
map\ldots{}

In order to make these maps actually winnable we would have to put in
some logic so that, for example, the blocks/platforms aren't too far
apart for the sprite to jump to/from.

But at least it's a start!


    % Add a bibliography block to the postdoc
    
    
    
    \end{document}
